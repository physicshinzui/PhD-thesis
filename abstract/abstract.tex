\chapter*{Abstract}
\section*{Introduction}
Protein functions are related to protein's own well-defined shapes/structures, whereas the functions are also linked to intrinsically disordered regions (IDRs), which are defined as regions of a protein unstructured in physiological conditions. Such proteins including IDRs are called intrinsically disordered proteins (IDPs).

p53 protein is known as an IDP. The protein activates transcription of target genes, which are involved in apoptosis, cell cycle arrest, and DNA repair, and hence is also known as a transcriptional factor. 
The transcriptional activation is regulated with the disordered region of p53, C-terminal domain (CTD), via multi-target recognition of CTD. The recognition is tuned by post-translational modifications (PTMs) on CTD, which may affect {\color{red}structural stability @FIXME} of CTD itself. For instance, the PTM, acetylation, is added to K382 of CTD, switching an interaction partner of CTD. However, the effects on structural stability have been unveiled.

One of CTD’s target proteins is S100B, which inhibits p53-dependent transcriptional activation and is used as a diagnostic marker of cancer. The S100B-CTD complex (PDBID: 1DT7) has been determined by Nuclear Magnetism Resonance (NMR). In this complex, CTD forms the helical structure on a hydrophobic region in S100B. It is, however, unclear that a variety of binding modes of CTD to S100B.

This study consists of two parts: (1) As a model of PTMs, I have investigated acetylation effects on CTD’s conformational ensemble [1]. (2) I have examined molecular recognition mechanisms of CTD through a variety of binding modes of CTD on S100B [In preparation].

\section*{Methods}
I performed all-atom generalised ensemble molecular dynamics (MD) simulations, virtual-system coupled Multicanonical MD (V-McMD) simulations. V-McMD enables a biomolecular system to sample a variety of conformations. S. Iida et al. have, indeed, shown its effectiveness for a system including two Endthelin-1 derivative monomers (See section \ref{et1_sec}, p\pageref{et1_sec}).

	In this study, V-McMD simulations with explicit water molecules were performed for the following systems: (1) An acetylated or non-acetylated CTD fragment. (2) A S100B monomer and a CTD fragment.

\section*{Results and Discussion}
(1) First, I have found that the both systems form various configurations. This result corresponds to the fact that CTD is disordered. 
Second, I have shown that the acetylation varies CTD’s conformational ensemble. 
For the two reasons, I suggest that the structural variety and the variation modulate CTD’s multiple target recognition. 
Fourth, I have demonstrated that the S100B-bound conformation is contained in the ensemble, and hence I suggest that the bound structure exists even in CTD’s isolated states, and it is used to bind to S100B.  Third, in order to validate the results from biochemical viewpoints, I also conducted circular dichroism (CD) spectroscopy measurements for a CTD fragment in the presence of and in the absence of acetylation. I have illustrated that results obtained computationally correspond qualitatively to those obtained from the CD measurements, or specifically, that the acetylation, both ways, tends to enhance helical conformations of CTD.
 
(2) First, I have indicated that CTD forms a variety of binding modes to S100B, which means that CTD forms fuzzy complexes. 
Second, I have shown that the native-like complex is the most free-energetically stable. This result validates the simulation. 
Third, I have found that each free-energetically stable state of binding modes is connected by low free energy barriers, and hence I conclude that each binding mode interconverts even on a surface of S100B. 
Fourth, I have demonstrated that conformations of an isolated CTD are contained in the conformational ensemble of CTD with S100B. 
This indicates that conformations of an isolated CTD are supplied as bound structures when CTD binds to S100B.

I have provided atomic insights into molecular recognition mechanisms and regulation by PTMs for the IDR, CTD. I suppose that the insights boost not only fundamental understanding of IDPs but also structure-based drug design targeting to IDPs/IDRs.
